% !LW recipe=pdflatex
% !TEX root = tvuontis.tex
%% This file is modified by Jussi Kangasharju and Pirjo Moen.
%% Earlier versions were made by Veli Mäkinen
%% from HY_fysiikka_LuKtemplate.tex authored by Roope Halonen ja
%% Tomi Vainio. Some text is also inherited from engl_malli.tex by
%% Kutvonen, Erkiö, Mäkelä, Verkamo, Kurhila, and Nykänen.
%% 
%% 
% STEP 1: Choose oneside or twoside
\documentclass[english,oneside,openany]{UH_DS_report}
%finnish,swedish
%
%\usepackage[utf8]{inputenc} 
% For UTF8 support. Use UTF8 when saving your file.
\usepackage{lmodern} % Font package 
\usepackage{textcomp} % Package for special symbols 
\usepackage[pdftex]{color, graphicx} % For pdf output and jpg/png graphics 
\usepackage[pdftex, plainpages=false]{hyperref} % For hyperlinks and pdf metadata 
\usepackage{fancyhdr} % For nicer page headers 
\usepackage{tikz} % For making vector graphics (hard to learn but powerful)
%\usepackage{wrapfig} % For nice text-wrapping figures (use at own discretion)
\usepackage{amsmath, amssymb} % For better math
%\usepackage[square]{natbib} % For bibliography
\usepackage[footnotesize,bf]{caption} % For more control over figure captions 
\usepackage{blindtext} 
\usepackage{titlesec}
\usepackage[titletoc]{appendix}

\onehalfspacing %line spacing
%\singlespacing 
%\doublespacing

%\fussy 
\sloppy % sloppy and fussy commands can be used to avoid overlong text lines

% STEP 2: 
% Set up all the information for the title page and the abstract form. 
% Replace parameters with your information.
\title{VMBC Report} 
\author{Tuomas Vuontisjarvi}

\date{\today}
\keywords{layout, summary, list of references} 
\depositeplace{}
\additionalinformation{}

% \classification{\protect{\ \\
%     \  General and reference $\rightarrow$ Document types $\rightarrow$ Surveys and overviews\  \\
%     \  Applied computing $\rightarrow$ Document management and text processing $\rightarrow$ Document management $\rightarrow$ Text editing\\
% }}

% If you want to quote someone special. You can comment this line and
% there will be nothing on the document.
%\quoting{Bachelor's degrees make pretty good placemats if you get them
%laminated.}{Jeph Jacques}

% OPTIONAL STEP: Set up properties and metadata for the pdf file that
% pdfLaTeX makes. But you don't really need to do this unless you want
% to.
\hypersetup{ 
	%bookmarks=true,         % show bookmarks bar first?
	unicode=true,           % to show non-Latin characters in Acrobat's bookmarks 
	pdftoolbar=true,        % show Acrobat's toolbar?
	pdfmenubar=true,        % show Acrobat's menu? 
	pdffitwindow=false,		% window fit to page when opened 
	pdfstartview={FitH},    % fits the width of the page to the window 
	pdftitle={},            % title
	pdfauthor={},           % author 
	pdfsubject={},          % subject of the document 
	pdfcreator={},          % creator of the document
	pdfproducer={pdfLaTeX}, % producer of the document
	pdfkeywords={something} {something else}, % list of keywords for
	pdfnewwindow=true,      % links in new window 
	colorlinks=true, 		% false: boxed links; true: colored links 
	linkcolor=black,        % color of internal links 
	citecolor=black,        % color of links to bibliography 
	filecolor=magenta,      % color of file links 
	urlcolor=cyan			% color of external links
}
          
\begin{document}

% Generate title page.
\maketitle

% STEP 3: Write your abstract (of course you really do this last). You
% can make several abstract pages (if you want it in different
% languages), but you should also then redefine some of the above
% parameters in the proper language as well, in between the abstract
% definitions.

\begin{abstract}
This report is about Variational Bayesian Monte Carlo (VBMC), a method for performing 
Bayesian inference with complex and computationally expensive black-box models. Key concepts
related to the VMBC are explained to provide clear understanding of understanding. With those 
definitions explained, this report will explore the use cases for the algorithm and cases
were the algorithm is unfit for, or if there are better methods available. Finally the usage
of the algorithm is explored with the provided pyVMBC python package.
\end{abstract}

% Place ToC
\mytableofcontents

\mynomenclature

% ----------------------------------------------------------------------
% STEP 4: Write the report. Your actual text starts here.
% You shouldn't mess with the code above the line except to change the
% parameters. Removing the abstract and ToC commands will mess up stuff.

\chapter{Introduction}
\label{chapter:intro}

According to Acerbi (2018), a significant problem with probabilistic models that have expensive,
black-box likelihoods is that the characteristics prevent the usage of standard techniques for 
Bayesian inference.

In order to address the problem of high computational cost, a novel sample efficient method 
has been introduced, called Variational Bayesian Monte Carlo (VMBC). 
Acerbi \cite{acerbi2018} claims that the VMBC solves the previously costly problem by combining 
variational inference with Bayesian quadrature, solving model posteriors efficiently 
and with a relatively small amount of sampling.

This report aims to explain the VMBC by exploring the key concepts behind it. 
Chapter three focuses on explaining all the relevant concepts and providing examples 
and on the way. The goal is to build a clear picture of the background domain 
and the issues researchers face.

Chapter four will explore the VMBC in more detail with the advantage 
of clearly defined concepts. The aim is to explain how it works and 
what kind of problems it solves. Also the chapter will consider problems that 
the VMBC is ill suited for.

Chapter five will dive deeper into the workings of the 
VMBC by using the python package pyVBMC provided by the author in another paper. 

\chapter{Key concepts and VMBC explained}
\label{chapter:structure}
\begin{enumerate}
  \renewcommand{\labelenumi}{\roman{enumi}.}
  \item Black-box models
  \item Bayesian inference
  \item Approximate inference methods
  \item Model posteriors
  \item Gaussian process
  \item Monte Carlo algorithms
\end{enumerate}

% The language around VMBV can be slightly confusing, mostly due to the 
% fact that the word model is used a lot and in a few different ways. 
% There are model posteriors, black-box models, model parameters and model instantiated hypotheses. 
% In order to avoid confusion around the issue, this report begins by defining key concepts.

The VMBC is used for complex and computationally expensive black-box models. 
Acerbi \cite{acerbi2018}\cite{acerbi2020}
notes a few examples of such models, such as computational neuroscience, biology and big data models.
The algorithm is a novel approximate inference  method for learning about black-box models. A model is black-box model when there is no access to its inner processes. This means that the 
model can be viewed completely in terms of its inputs and outputs. 

One way of learning about the model is Bayesian inference, which is a method for computing posterior distribution
over parameters and the model evidence. However, since Bayesian inference is 
generally analytically intractable \cite{acerbi2018}, statistical approximate inference methods 
are often used. These methods include Markov Chain Monte Carlo algorithms and variational inference.

As Acerbi \cite{acerbi2018} notes, existing methods of approximate inference, 
such as above examples, generally require knowledge about the 
model in order to produce approximate inference. When a method requires more knowledge about the model
than just inputs and outputs, by definition it can't be applied to black-box models. Some methods
can bypass this requirement when given a very large number of model evaluations.

A computationally expensive black-box model is a model where evaluating the model is time consuming, 
which means that there generally isn't access to large number of model evaluations. Therefore the existing methods for approximate Bayesian inference are unfit for 
computationally expensive black-box models. Expensive model is defined as one evaluation taking 
one second or more per evaluation\cite{pyvmbc}

From the point of view of the pyVBMC, a python package for performing VMBC model 
and posterior inference, the black-box model is provided as Python function, 
which calculates target log likelihood of the black-box model.

The VMBC produces a flexible approximate posterior distribution of the model parameters 
\cite{acerbi2018}. The posterior distribution is a joint probability distribution which describes 
how plausible each parameter is given the observed data. The posterior is expressed as $p(x \mid \mathcal{D})$,
where $\mathcal{D}$ is the dataset or the evidence and $x \in \mathbb{R}^D$, 
which are the black-box model parameters.

The black-box model for the algorithm is expressed as $p(\mathcal{D} \mid x)$. This means that the 
input for the algorithm is a likelihood function which provides the log-likelihood for seeing the 
evidence with the given priors.


\chapter{VMBC use cases and fallouts}
\label{chapter:layout}

\chapter{pyVMBC examples}
\label{chapter:layout}

\chapter{Conclusions}
\label{chapter:conclusions}


% STEP 5: Uncomment the following lines and set your .bib file and
% desired bibliography style to make a bibliography with BibTeX.
% Alternatively you can use the thebibliography environment if you want
% to add all references by hand.
% 
\cleardoublepage %fixes the position of bibliography in bookmarks
\phantomsection

\renewcommand\bibname{References}
\addcontentsline{toc}{chapter}{\bibname} % This lines adds the bibliography to the ToC 
\bibliographystyle{abbrv} % numbering alphabetic order 
\bibliography{UH_DS_references}


\end{document}
